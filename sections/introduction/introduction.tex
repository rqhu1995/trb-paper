\section{Introduction}
The \citep{TRBGuide} has unique and somewhat arbitrary requirements for papers submitted for review and publication. While the initial submission is required to be in PDF format, submissions for publication in Transportation Research Record must be in Microsoft Office format. On top of this, the manuscripts must be line-numbered, captions are bolded and employ atypical punctuation, and the references must be numbered when cited and then printed in order. More details about the manuscript details can be found online at~\url{http://onlinepubs.trb.org/onlinepubs/AM/InfoForAuthors.pdf}.

It is assumed that the readers of this document have some significant level of experience in \LaTeX~and \verb1bibtex1. As use of literate programming becomes more widespread in engineering and planning, it is possible that this template may need to be made more robust.

% Example of how to include figures in this section:
% \begin{figure}[!ht]
%   \centering
%   \includegraphics[width=0.6\textwidth]{sections/introduction/figures/sample-figure}
%   \caption{This is a sample figure in the introduction section.}\label{fig:intro-sample}
% \end{figure}

\subsection{History}
David Pritchard posted the original versions of this template in 2009 and updated it in 2011, soon after TRB began allowing PDF submissions. Gregory Macfarlane made significant adaptations to it in March 2012, allowing for Sweave integration and automatic word and table counts. Ross Wang automated the total word count and made some formatting modifications in July 2015. Version 2.1.1 has been made available on GitHub in January, 2016.  Version 3.1 has been made available on Github (\url{https://github.com/chiehrosswang/TRB_LaTeX_rnw}) in June, 2017.  Versions 2.1.1 Lite and 3.1 Lite were made available on GitHub (\url{https://github.com/chiehrosswang/TRB_LaTeX_tex}) in June, 2017 for users who do not need R and Sweave functions provided in the original verions.